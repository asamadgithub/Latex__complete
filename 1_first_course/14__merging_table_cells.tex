\documentclass{article}


\usepackage{multirow}




\begin{document}
	
	% -------------------------------------------------------------------------------------------------------------
	\section{Multi-Row Table without lines}
     Here we will see, how to merge many of the cells in the document.
     
     
		\begin{table}[h]
			\begin{tabular}{cccccccc}
		  
\multirow{4}{*}{Surface}       & \multirow{4}{*}{Material} & \multicolumn{6}{c}{Hardness}   \\   
		                                    &                                 & \multicolumn{2}{c}{Soft} & \multicolumn{2}{c}{Medium} & \multicolumn{2}{c}{Hard}  \\ 
		                                    &                                 & \multicolumn{6}{c}{Factors}                                                                        \\  
		                                    &                                 &      A      &   B         &     A       &    B        &     A      &   B    \\     
\multirow{2}{*}{Base Metal}  &		  Cast Iron           &      10     &   20       &   30       &    40      &    50     &     60  \\   
		                                    &           Steel              &      70     &   80       &   90       &   100     &   110      &   120    \\  
\multirow{2}{*}{Dressing}      & 		 Cast Iron          &       10     &     20     &    30      &     40    &    50      &  60     \\   
		                                    &            Steel             &       70    &     80     &      90    &    100     &     110   &   120    \\  
\multirow{2}{*}{S Dressing}  & 		 Cast Iron          &       10     &       20   &    30      &    40      &     50    &   60    \\	 
                                            &            Steel             &      70      &     80     &     90     &      100  &   110     &   120    \\                 
		  
		    \end{tabular}
		\caption{Table of factor for use in speed formulas}	
		\end{table}	
	
	% -------------------------------------------------------------------------------------------------------------
	
	
		\section{Multi-Row Table WITH LINES}
	Here we will see, how to merge many of the cells in the document.
	
	
	\begin{table}[h]
		\centering
		\begin{tabular}{|c|c|c|c|c|c|c|c|}   \hline
			
			\multirow{4}{*}{Surface}       & \multirow{4}{*}{Material} & \multicolumn{6}{c|}{Hardness}   \\   \cline{3-8}
			&                                 & \multicolumn{2}{c|}{Soft} & \multicolumn{2}{c|}{Medium} & \multicolumn{2}{c|}{Hard}  \\  \cline{3-8}
			&                                 & \multicolumn{6}{c|}{Factors}                                                                        \\      \cline{3-8}
			&                                 &      A      &   B         &     A       &    B        &     A      &   B    \\     \hline
			\multirow{2}{*}{Base Metal}  &		  Cast Iron           &      10     &   20       &   30       &    40      &    50     &     60  \\  \cline{2-8} 
			&           Steel              &      70     &   80       &   90       &   100     &   110      &   120    \\     \hline
			\multirow{2}{*}{Dressing}      & 		 Cast Iron          &       10     &     20     &    30      &     40    &    50      &  60     \\   \cline{2-8}
			&            Steel             &       70    &     80     &      90    &    100     &     110   &   120    \\    \hline
			\multirow{2}{*}{S Dressing}  & 		 Cast Iron          &       10     &       20   &    30      &    40      &     50    &   60    \\	 \cline{2-8}
			&            Steel             &      70      &     80     &     90     &      100  &   110     &   120    \\            \hline
			
		\end{tabular}
		\caption{Table of factor for use in speed formulas}	
	\end{table}	
	
	
	
	
	
	
		\begin{table}[h]
		\centering	
		\begin{tabular}{ccccc}   
			\hline \hline
			    & \multicolumn{2}{c}{GGA} & \multicolumn{2}{c}{LDA} \\
	     	Reconstruction	& Core d & Valence d & Core d & Valence d   \\ \hline 
	     	(4$\times$1)	& 0  &  5 & 1 & 2 \\
	     	(8$\times$1)	& -0.4  &  -5 & -0.5 & -0.6 \\
	     	(4$\times$2)	& 36  &  48 & 5 & 15 \\
	     	(8$\times$2)	& 25  &  27 & -12 & 2 \\ \hline \hline
		\end{tabular}
		\caption{Table of factor for use in speed formulas}	
	\end{table}	
	
	
	
	
	
	
	
	
	% -------------------------------------------------------------------------------------------------------------
	
	
	
	
	
\end{document}
\documentclass[a4paper,12pt]{article}

\usepackage[text={6in,8in}]{geometry}
\usepackage[inline]{enumitem}

\title{Get to know the lists in Latex}
\author{Maxi}
\date{Oct 18th, 2019}

\begin{document}
	\maketitle
	There are three list environment, namely: 1): itemize , 2): enumerate , 3): description \\
	They are included in the package called : package  "enumitem" as above 
	
	%-----------------------------------------------------------------
	\section{Vertical List in Latex}
		\subsection{Itemize Environment in Latex List}
		
			\begin{itemize}[leftmargin=3in, label = $\ast$]
				\item List1
				\item List2
				\item List3
			\end{itemize}
		
		\subsection{Enumerate Environemnt in Latex List}
			\begin{enumerate}[label=\Roman*]   %roman,alph,arabic..
				\item The first Enu
				\item The second Enu
			\end{enumerate}
		
		Here is some text of paragraph with the intention that later we will restart from the "Fourth" Enu so
		
		\begin{enumerate}[resume,label=\Roman*]   %roman,alph,arabic..
			\item: The fourth Enu
			\item The fifth Enu
			\item The sixth Enu
			\subitem This is something I want to exclude from the item but need to insert inbetween.
			\item The seven Enu
		\end{enumerate}
	
	\subsection{How to Put a box around these number}
		\begin{enumerate}[label=\fbox{\arabic*}]
			\item Box
			\item Box
		\end{enumerate}
	
	
	%-----------------------------------------------------------------
	\section{Horizontal List}
		\begin{enumerate*}
			\item one one one
			\item two two two
			\item three three three
		\end{enumerate*}
	%-----------------------------------------------------------------
	\section{Different Marking symbol for each element}
		\begin{enumerate}
			\item[$-$]        Minues item
			\item[$\ast$]    asteric item
			\item[$\star$]   star item
		\end{enumerate}
		%-----------------------------------------------------------------
	\section{Description}
		\begin{description}
			\item[First:] Can we look at the  description
			\item[Second:] Yes we "CAN" look at the  description
	    \end{description}
		

\end{document}



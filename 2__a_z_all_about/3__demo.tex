\documentclass{book}
\usepackage{amsmath}
\usepackage{ esint }


\begin{document}
\title{Typesetting Mathematics in \LaTeX}
\author{Nauman \\ recluze@gmail.com}
\maketitle


\chapter{Introduction} 
\LaTeX\ is extremely powerful when it comes to typesetting mathematics. It's one of the core strengths of this system. 

\section{Displaying Mathematics}
There are two ways of displaying maths. One is \emph{inline} and the other is \emph{display} format -- in which the whole math sits on its own set of lines.


\subsection{Inline Mode}
We are going to insert a mathematics equation inline here using a pair of \$ signs:  $E = mc^{12} $  . As you can see, the display (such as line spacing) does not get messed up by the mathematics as it does with word processing softwares. 

\subsection{Display Mode}
We can also display equations in their own set of lines. To do this, we can use the equation environment. 

\begin{equation}\label{eq:emc}
E=mc^2
\end{equation}

As you can see, \LaTeX\ inserts the equation number automatically. We can refer to it using the \verb|\ref| command just as we referred to sections, figures and tables. (E.g. Equation~\ref{eq:emc}.) To get rid of the equation number, simply use the \emph{star variant} of the equation environment. (For this, you need the \texttt{amsmath} package.)

\begin{equation}\label{eq:pvnrt}
PV=nRT
\end{equation}

Equation~\ref{eq:pvnrt} tells us the ideal gas relationship.

Alternatively, we can use the shorthand keys \verb|\[| and \verb|\]|

% -------------------------------------------------------------------

\chapter{Writing some equation in Book class for practise}

Here I will write Some of the Mathematical equation

\begin{equation}
\sqrt{\frac{x}{\frac{y}{z}}}
\end{equation}







% -------------------------------------------------------------------

\section{Mathematical Features}
\LaTeX\ has many builtin features and you can get many more easily. Here, we'll see some of these features: 

Addition, subtraction, multiplication and division: 


Superscripts and subscripts: 



Summation, union, intersection, big-union, integral: 

\begin{equation}\label{eq:summ}
\sum_{i=1}^{n} \sum_{j=2}^{m} X(i,j)
\end{equation}

\begin{equation}\label{eq:union}
 x \cup y \cap z
\end{equation}


\begin{equation}\label{eq:union}
X \bigcup Y \bigcap Z
\end{equation}


\begin{equation}\label{eq:bigcup}
\bigcup_{i=0}^{m} x_i
\end{equation}

\begin{equation}\label{eq:int1}
\int_0^1 x_i^2
\end{equation}



Fractions, brackets, square root: 

\begin{equation}\label{eq:nested_frac}
\frac{
	\sum_{i}^{n} x_1^{23}
}{
	\int_{65}^{25} x^{2}}
\end{equation}

Greek letters: 

\begin{equation}\label{eq:nested_frac}
\alpha_2 + \beta^2 + \gamma + \Gamma + \theta   + \Theta  + \epsilon
\end{equation}

Matrices and vectors. For this, you need to include the \texttt{amsmath} package and then use the \texttt{bmatrix} or \texttt{pmatrix} environment: 


\begin{equation}\label{eq:matrix}
	\begin{pmatrix}
    1    &    2   &   3   \\
    4   &    5   &   6
	\end{pmatrix}
\end{equation}



Accents: 

\begin{equation}\label{eq:hat1}
\hat{x} + \hat{\imath} + \hat{i}  + \dot{x}
\end{equation}




See the \texttt{Math} menu in the IDE for other operations. You can refer to ``Short Math Guide for \LaTeX'' for a lot more examples. 

\section{Using Symbols}
You might come across situations where you need to find new symbols. For this, you can refer to the ``The Comprehensive \LaTeX Symbols List''.  

\begin{equation}\label{eq:hat1}
\oint  +   \varointctrclockwise + \sqint
\end{equation}


(Optional) Since this is a long command, we might want to create a shortcut using the \verb|\newcommand| command in the preamble. This also allows us to later change the symbol without having to change the equations. 



\end{document}